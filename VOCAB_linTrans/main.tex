\documentclass{ximera}
%% You can put user macros here
%% However, you cannot make new environments

\graphicspath{
    {./}
    {ABOUT/}
    {./LTR-0070/}
    {./VEC-0060/}
    {./APP-0020/}
    {./HowToUse/}
    {./APP-0070/}
    {./pictures/}
    {./PROB_linTrans}
    }

\usepackage{tikz}
\usepackage{tkz-euclide}
\usepackage{tikz-3dplot}
\usepackage{tikz-cd}
\usetikzlibrary{shapes.geometric}
\usetikzlibrary{arrows}
%\usetkzobj{all}
\pgfplotsset{compat=1.13} % prevents compile error.

%\renewcommand{\vec}[1]{\mathbf{#1}}
\renewcommand{\vec}{\mathbf}
\newcommand{\RR}{\mathbb{R}}
\newcommand{\dfn}{\textit}
\newcommand{\dotp}{\cdot}
\newcommand{\id}{\text{id}}
\newcommand\norm[1]{\left\lVert#1\right\rVert}
 
\newtheorem{general}{Generalization}
\newtheorem{initprob}{Exploration Problem}

\tikzstyle geometryDiagrams=[ultra thick,color=blue!50!black]

%\DefineVerbatimEnvironment{octave}{Verbatim}{numbers=left,frame=lines,label=Octave,labelposition=topline}



\usepackage{mathtools}


\title{Essential Vocabulary} \license{CC BY-NC-SA 4.0}



\begin{document}
\begin{abstract}
\end{abstract}
\maketitle


\begin{onlineOnly}
\section*{Essential Vocabulary}
Here is a  \href{https://quizlet.com/881362019/chapter-6-vocabulary-flash-cards/?i=y06sd&x=1jqt}{link to a list of these terms on Quizlet}
\end{onlineOnly}

\begin{tikzpicture}[scale=1]
   \filldraw[teal, opacity=0.3](0,0)--(20,0)--(20,0.1)--(0,0.1)--cycle;
 \end{tikzpicture}

Composition of linear transformations
\begin{expandable}{}{}
    Let $U$, $V$ and $W$ be vector spaces, and let $T:U\rightarrow V$ and $S:V\rightarrow W$ be linear transformations.  The \dfn{composition} of $S$ and $T$ is the transformation $S\circ T:U\rightarrow W$ given by
$$(S\circ T)(\vec{u})=S(T(\vec{u}))$$

The matrix of a composition is the product of the matrices corresponding to the transformations in the composition, in the same order.
\end{expandable}

\begin{tikzpicture}[scale=1]
   \filldraw[teal, opacity=0.3](0,0)--(20,0)--(20,0.1)--(0,0.1)--cycle;
 \end{tikzpicture}

Image of a linear transformation
\begin{expandable}{}{}
    Let $V$ and $W$ be vector spaces, and let $T:V\rightarrow W$ be a linear transformation.  The \dfn{image} of $T$, denoted by $\mbox{im}(T)$, is the set
$$\mbox{im}(T)=\{T(\vec{v}):\vec{v}\in V\}$$
In other words, the image of $T$ consists of individual images of all vectors of $V$.
\end{expandable}

\begin{tikzpicture}[scale=1]
   \filldraw[teal, opacity=0.3](0,0)--(20,0)--(20,0.1)--(0,0.1)--cycle;
 \end{tikzpicture}

Inverse of a linear transformation
\begin{expandable}{}{}
    Let $V$ and $W$ be vector spaces, and let $T:V\rightarrow W$ be a linear transformation.  A transformation $S:W\rightarrow V$ that satisfies $S\circ T=\id_V$ and $T\circ S=\id_W$ is called an \dfn{inverse} of $T$. If $T$ has an inverse, $T$ is called \dfn{invertible}.
\end{expandable}

\begin{tikzpicture}[scale=1]
   \filldraw[teal, opacity=0.3](0,0)--(20,0)--(20,0.1)--(0,0.1)--cycle;
 \end{tikzpicture}

Kernel of a linear transformation
\begin{expandable}{}{}
    Let $V$ and $W$ be vector spaces, and let $T:V\rightarrow W$ be a linear transformation.  The \dfn{kernel} of $T$, denoted by $\mbox{ker}(T)$, is the set
$$\mbox{ker}(T)=\{\vec{v}:T(\vec{v})=\vec{0}\}$$
In other words, the kernel of $T$ consists of all vectors of $V$ that map to $\vec{0}$ in $W$.
\end{expandable}

\begin{tikzpicture}[scale=1]
   \filldraw[teal, opacity=0.3](0,0)--(20,0)--(20,0.1)--(0,0.1)--cycle;
 \end{tikzpicture}

Linear transformation
\begin{expandable}{}{}
    A transformation $T:\RR^n\rightarrow \RR^m$ is called a \dfn{ linear transformation} if the following are true for all vectors $\vec{u}$ and $\vec{v}$ in $\RR^n$, and scalars $k$.
$$T(k\vec{u})= kT(\vec{u})$$

$$T(\vec{u}+\vec{v})= T(\vec{u})+T(\vec{v})$$
\end{expandable}

\begin{tikzpicture}[scale=1]
   \filldraw[teal, opacity=0.3](0,0)--(20,0)--(20,0.1)--(0,0.1)--cycle;
 \end{tikzpicture}

Standard matrix of a linear transformation
\begin{expandable}{}{}
    Let $T:\RR^n\rightarrow\RR^m$ be a linear transformation.  Then the matrix
 
 $$A=\begin{bmatrix}
           | & |& &|\\
		T(\vec{e}_1) & T(\vec{e}_2)&\dots &T(\vec{e}_n)\\
		|&| & &|
         \end{bmatrix}
$$
is known as the \dfn{standard matrix of the linear transformation} $T$.
\end{expandable}

\begin{tikzpicture}[scale=1]
   \filldraw[teal, opacity=0.3](0,0)--(20,0)--(20,0.1)--(0,0.1)--cycle;
 \end{tikzpicture}



\end{document}
