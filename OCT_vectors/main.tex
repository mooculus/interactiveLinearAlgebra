\documentclass{ximera}
\input{../preamble.tex}

\title{Octave for Chapter 1} \license{CC BY-NC-SA 4.0}
\begin{document}
\begin{abstract}
\end{abstract}
\maketitle
\section*{Octave for Chapter 1}

The templates in this section provide sample Octave code for vector operations. You can access our code through the link at the bottom of each template.  Feel free to modify the code and experiment to learn more!  

You can write your own code using Octave software or online Octave cells.  To access Octave cells online, go to the \href{https://sagecell.sagemath.org/}{Sage Math Cell Webpage}, select OCTAVE as the language, enter your code, and press EVALUATE.  

To ''save" or share your online code, click on the \emph{Share} button, select \emph{Permalink}, then copy the address directly from the browser window.  You can store this link to access your work later or share this link with others.  You will need to get a new Permalink every time you modify the code.

\subsection*{Octave Tutorial}
\subsubsection*{Basic Operations}

\begin{template}\label{temp:vectorOps}
Here we present basic vector operations.
\begin{verbatim}
    % Define vectors v and w
    v=[3;1;1;2;4];
    w=[-2;1;0;3;-1]; 

    % Find v+w  
    sum=v+w

    % Find 5v
    five_v=5*v

    % Find the dot product of v and w
    dot_v_w=dot(v,w)

    % Find the norm of v
    norm_v=norm(v)
\end{verbatim}

\href{https://sagecell.sagemath.org/?z=eJxdTUsKwyAQ3QveYTaBpG2gSZqVuCu9RChSolIX0WDMeP2O3TSUWbw37zNTwd1Y5w2gmVOIGyC8vIbMGcppEB1NL25PwVmWU9vTehWDaLuicFbBw1EazxmAs21fJNGDMSJn1qFRKMcTHoz0NqBDgjUGvc8Jgv09Jl2hypKwxktu_mo-xOWb56xQOl2gxuYDFmw51w==&lang=octave&interacts=eJyLjgUAARUAuQ==}{Link to code}.

\begin{remark}
    Note that lines 2 and 3 of our code have a semi-column at the end while lines 6, 9, 12, and 15 do not.  Remove the semi-column from lines 2 and 3, add semi-columns to the other lines.  What happens?
\end{remark}

\begin{verbatim}
    % Define vectors v and w
    v=[3;1;1];
    w=[-2;1;0];
    
    % Find the cross product v x w
    cross_v_w=cross(v,w)
    \end{verbatim}
    
    \href{https://sagecell.sagemath.org/?z=eJxTVXBJTcvMS1UoS00uyS8qVihTSMxLUSjn5SqzjTa2NrQ2jLXm5Sq3jdY1AnIMQBxeLlUFt0ygopKMVIXkovziYoWCovyU0uQSoOYKkFawYHxZfLktmKVRplOuCQBmGSA1&lang=octave&interacts=eJyLjgUAARUAuQ==}{Link to code}.

\end{template}

\begin{example}\label{ex:angleInOctave}
    Find the angle between vectors $\vec{v}$ and $\vec{w}$ from the previous example.  (Note the syntax for the $\arccos{}$ function.)
    \begin{explanation}
        \begin{verbatim}
% Define vectors v and w
v=[3;1;1;2;4];
w=[-2;1;0;3;-1];

% Find the angle between v and w in radians and degrees
angle_radians = acos(dot(v,w)/(norm(v)*norm(w)))
angle_degrees = rad2deg(angle_radians)
        \end{verbatim}

\href{https://sagecell.sagemath.org/?z=eJxVjM8KAiEQxu-C7zCXBY2WWreb7C16iSXC1mkTSkFFX79pyUPMYfj-_L4OzvhwHqHgkkNMUMB4C5WzMs2jHuiUPl01Z3Wae0XyqEfdD1-Hsw4ujtr5iUStL4Q75oro2wo4D9FYZ3zaDItrREycbe1biyYwS0jChizKvsqD8CG-RZG77VcpZSN-A0QQq0iJvyX5AVWtQWc=&lang=octave&interacts=eJyLjgUAARUAuQ==}{Link to code}.        
    \end{explanation}
\end{example}

Sometimes it is useful to reference a single component of a vector.
\begin{template}\label{temp:vectorComp}
    Given vector $\vec{v}=\begin{bmatrix}1\\-2\\-5\end{bmatrix}$, let's find the third component of $\vec{v}$.
\begin{verbatim}
% Define vector v
v=[1;-2;-5];

% Find the third component of v
v_3=v(3)
\end{verbatim}

    \href{https://sagecell.sagemath.org/?z=eJxTVXBJTcvMS1UoS00uyS9SKOPlKrONNrTWNbLWNY215uXi5VJVcMvMS1EoyUgF4syiFIXk_NyC_LzUvBKF_DSwhnhj2zINY00AVTYWYw==&lang=octave&interacts=eJyLjgUAARUAuQ==}{Link to code}.
\end{template}

\subsubsection*{Loops}
In this section we will introduce one of the most fundamental concepts in programming, a \emph{loop}.  Loops are used to generate sequences of objects (such as components of a vector) according to some rule.  We will start with a simple \emph{for} loop.  Every time you perform the steps inside the loop, you come back to the beginning, and increase the index $i$ by 1 until you reach the desired number of iterations.  Try to guess the outcome of the loop in each example, then run the code to verify your guess.

\begin{example}\label{ex:loop1}
        \begin{verbatim}
% Enter the starting value
n=10;

% At every iteration, n gets replaced with n+1
for i=1:10
    n=n+1
end
    \end{verbatim}

\href{https://sagecell.sagemath.org/?z=eJwdyjEOglAQRdF-ktnDa6iw4LeSX1i4kB95wiRkMMOIcfcSb3lyO9w9GciF2LNFms842vqmitcyjCoqHW4JHowv7Jxb2uYXOGbmjuBrbQ9O-Fgu8L6oPLeA1XItgwrOvP6ZPv0AoF8hKw==&lang=octave&interacts=eJyLjgUAARUAuQ==}{Link to code}.    
\end{example}

\begin{example}\label{ex:loop2}
    \begin{verbatim}
% Enter the first component of vector v
v(1)=1;

% Enter the second component of vector v
v(2)=1;

% we continue to assign values to vector components
for i=1:10
    v(i+2)=v(i)+v(i+1);
    v(i) % this prints each component individually
end
    \end{verbatim}

\href{https://sagecell.sagemath.org/?z=eJx1jjEKwzAMRXeD76AlEJOl7tiQsQcxttwIUjnYiktvXwdKmqVapC_pfX4HdxbMIDNCpFwEfHquiZEFUoSKXlKGqlXtrZnsqJVW3Ykp6BOH_9D1B72wvbEQbwiSwJVCD4bqlg3Lvvhih1XRKjZNk73Zi1bQqvY0NMfWzLDP1ozHwUDXElGBNVODAZ2fT7mIA1UKm1uWt1bI4QOvKk6t&lang=octave&interacts=eJyLjgUAARUAuQ==}{Link to code}.   

Do you recognize this famous sequence?
\end{example}

\subsection*{Octave Exercises}

\begin{problem}\label{prob_oct_vec_2}
    Use Octave to find the angle (in degrees and radians) between vectors 
    $$\vec{v}=\begin{bmatrix}2\\-4\\10\\-25\end{bmatrix},\quad\vec{w}=\begin{bmatrix}5\\3\\-20\\11\end{bmatrix}$$
\end{problem}

\begin{problem}\label{prob_oct_vec_3}
Write Octave code to show that the points $A(1, -2, 3, 5)$, $B(-2, 0, 1, 1)$, $C(2, 4, -3, 2)$ are the vertices of a right triangle in $\RR^4$.
\end{problem}

\begin{problem}\label{prob_oct_vec_1}
    Use vectors $\vec{v}$ and $\vec{w}$ from Template \ref{temp:cross} to illustrate that the cross product of two vectors is orthogonal to both vectors.  Pick another pair of vectors $\vec{v}$ and $\vec{w}$ and repeat the demonstration.
    \begin{hint}
        \begin{verbatim}
% Define vectors v and w
v=[3;1;1];
w=[-2;1;0];

% Find the cross product v x w
cross_v_w=cross(v,w)

% Check to see that the cross product is orthogonal to both v and w
dot(cross_v_w,v)
dot(cross_v_w,w)
\end{verbatim}

\href{https://sagecell.sagemath.org/?z=eJxtjsEKwjAQRO-B_sNeCg1UsHoMOSn-RJES09UEpSvJmvj5phUUxNvMMG9na9jj2U8ICS1TiJDATCPkSiTdb1WnuqOqRNb9alPMejaVqOHgS4kdgg0UI9wDjQ_LBX7O6BIOach6UU1qs3xzO4f2CkwQEQtv-M8RH4ECO7rQZG5z90Tsvn-NxM1noE3yN8nyBT-BRHk=&lang=octave&interacts=eJyLjgUAARUAuQ==}{Link to code}.
    \end{hint}
\end{problem}

\begin{problem}\label{prob_oct_vec_4}
    Write an Octave routine that allows the user to input three points and returns a normal vector to the plane containing these points.  Your code might start as follows:

    \begin{verbatim}
        % Enter coordinates of point A
        A=[1 4 6];
        
        % Enter coordinates of point B
        B=[-2 3 7]
        
        % Enter coordinates of point C
        C=[1 -2 4]
        
        % Normal vector to the plane containing A, B, C is given by
        
    \end{verbatim}

    What type of user inputs (other than all zeros) would result in the computed normal vector of $\vec{n}=\vec{0}$?  Explain this phenomenon geometrically.

    Use your code to find an equation of a plane containing points $(-2, 4, 0)$, $(-1, 8, 1)$, $(3, -1, 2)$.
\end{problem}    

\begin{problem}\label{prob_oct_vec_loop}
    Due to a false fire alarm, a programmer left line 10 of the following code unfinished.  Finish the code so that it computes the dot product (dp) of two vectors.  Note that this code uses Definition \ref{def:dotproduct} directly, instead of relying on the built-in \emph{dot} function.

    \begin{verbatim}
        % Define vectors v and w
        v=[3 6 -1 7 12];
        w=[-2 4 -3 5 11];

        % Initialize the dot product (dp)
        dp=0;

        % Find the dot product v*w directly from the definition
        for i=1:5
            dp=dp+;
        end

        % Print your answer
        dp

        % Check your answer
        correct_answer=dot(v,w)
    \end{verbatim}

    \href{https://sagecell.sagemath.org/?z=eJxdjssKwjAQRfeB_MPdFHwVjPUBSlaK4M59EZEmxWBNyhgT9OtNqRud3cy9nDkZdro2ViPoyjt6IOBiFSJnQZYFlsgFVhCz04azKMt8hjnyAgsI0Z04y3CwxptLY94a_qqhnEdLTj0rj4Fqh5ypVk6_3b1J8P9WGEUoQ0mgeaEmd-8bnVciO8tZ7QhGivWCM6RJQNWOE1Jb1XOPZKzHyz0p6T-ipu5rH22vurr9RpWj7tu532VyGYRJHH4ATLJPyQ==&lang=octave&interacts=eJyLjgUAARUAuQ==}{Link to code}.

\end{problem}
 
\end{document}