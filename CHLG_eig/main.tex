\documentclass{ximera}
%% You can put user macros here
%% However, you cannot make new environments

\graphicspath{
    {./}
    {ABOUT/}
    {./LTR-0070/}
    {./VEC-0060/}
    {./APP-0020/}
    {./HowToUse/}
    {./APP-0070/}
    {./pictures/}
    {./PROB_linTrans}
    }

\usepackage{tikz}
\usepackage{tkz-euclide}
\usepackage{tikz-3dplot}
\usepackage{tikz-cd}
\usetikzlibrary{shapes.geometric}
\usetikzlibrary{arrows}
%\usetkzobj{all}
\pgfplotsset{compat=1.13} % prevents compile error.

%\renewcommand{\vec}[1]{\mathbf{#1}}
\renewcommand{\vec}{\mathbf}
\newcommand{\RR}{\mathbb{R}}
\newcommand{\dfn}{\textit}
\newcommand{\dotp}{\cdot}
\newcommand{\id}{\text{id}}
\newcommand\norm[1]{\left\lVert#1\right\rVert}
 
\newtheorem{general}{Generalization}
\newtheorem{initprob}{Exploration Problem}

\tikzstyle geometryDiagrams=[ultra thick,color=blue!50!black]

%\DefineVerbatimEnvironment{octave}{Verbatim}{numbers=left,frame=lines,label=Octave,labelposition=topline}



\usepackage{mathtools}


\title{Challenge Problems} \license{CC BY-NC-SA 4.0}

\begin{document}

\begin{abstract}
\end{abstract}
\maketitle

\section*{Challenge Problems for Chapter 8}

\begin{problem}\label{prb:8.32} Suppose $A$ is an $n\times n$ matrix consisting entirely of real
entries but $a+ib$ is a complex eigenvalue having the eigenvector, $\vec{x}+i\vec{y}$ Here $\vec{x}$ and $\vec{y}$ are real vectors. Show
that then $a-ib$ is also an eigenvalue with the eigenvector, $\vec{x}-i\vec{y}$.

\begin{hint}
You should remember that the conjugate of a
product of complex numbers equals the product of the conjugates. Here $a+ib$
is a complex number whose conjugate equals $a-ib.$


Click the arrow to see the answer.
\begin{expandable}{}{}
$AX=\left(
a+ib\right)X$. Now take conjugates of both sides. Since $A$ is
real,
\[
A\overline{X}=\left( a-ib\right) \overline{X}
\]
\end{expandable}
 
\end{hint}
\end{problem}



\begin{problem}\label{prb:2x2diagonalizable}
If $A$ is $2 \times 2$ and diagonalizable, show that $C(A) = \{X \mid XA = AX\}$ has dimension $2$ or $4$. 
\begin{hint}
If $P^{-1}AP = D$, show that $X$ is in $C(A)$ if and only if $P^{-1}XP$ is in $C(D)$.
\end{hint}
\end{problem}

\begin{problem}\label{prob:real_ew_commuting}
Let $A$ be $n \times n$ with $n$ distinct real eigenvalues. If $AC = CA$, show that $C$ is diagonalizable.
\end{problem}

\begin{problem}\label{prob:similar_poly_eval}
Given a polynomial $p(z) = r_{0} + r_{1}z + \dots + r_{n}z^{n}$ and a square matrix $A$, the matrix $p(A) = r_{0}I + r_{1}A + \dots  + r_{n}A^{n}$ is called the \dfn{evaluation} of $p(z)$ at $A$. Let $B = P^{-1}AP$. Show that $p(B) = P^{-1}p(A)P$ for all polynomials $p(x)$.
\end{problem}

\begin{problem}\label{prb:diagonalizable_poly}
If $A$ is diagonalizable and $p(x)$ is a polynomial such that $p(\lambda) = 0$ for all eigenvalues $\lambda$ of $A$, show that $p(A) = O$ (here, the final O is the zero matrix the same size as $A$).

\begin{remark}
The characteristic polynomial of $A$ (see Definition \ref{def:char_poly_complex}) certainly satisfies the requirement that $p(\lambda) = 0$ for all eigenvalues $\lambda$ of $A$.  In solving this problem you have proved a special case of the Cayley-Hamilton theorem, see Theorem~\ref{th:Cayley_Hamilton}.  In fact, if $p(\lambda)$ is the characteristic polynomial of $A$, then $p(A) = 0$ whether or not $A$ is diagonalizable.
\end{remark}
\end{problem}

\begin{problem}\label{prob:5_5_12}
Let $P$ be an invertible $n \times n$ matrix. If $A$ is any $n \times n$ matrix, write $T_{P}(A) = P^{-1}AP$. Verify that:

\begin{enumerate}
\item\label{prob:5_5_12a} $T_{P}(I) = I$
\item\label{prob:5_5_12b} $T_{P}(AB) = T_{P}(A)T_{P}(B)$
%Solution $T_{P}(A)T_{P}(B) = (P^{-1}AP)(P^{-1}BP) = P^{-1}(AB)P = T_{P}(AB)$.
\item\label{prob:5_5_12c} $T_{P}(A + B) = T_{P}(A) + T_{P}(B)$
\item\label{prob:5_5_12d} $T_{P}(rA) = rT_{P}(A)$
\item\label{prob:5_5_12e} $T_{P}(A^{k}) = [T_{P}(A)]^{k}$ for $k \geq 1$
\item\label{prob:5_5_12f} If $A$ is invertible, $T_{P}(A^{-1}) = [T_{P}(A)]^{-1}$.
\item\label{prob:5_5_12g} If $Q$ is invertible, $T_{Q}[T_{P}(A)] = T_{PQ}(A)$.
\end{enumerate}

\end{problem}



\begin{problem}\label{prob:3x3_special_symmetric}
Let $A = \begin{bmatrix}
0 & a & b \\
a & 0 & c \\
b & c & 0	
\end{bmatrix}$ and $B =
\begin{bmatrix}
c & a & b \\
a & b & c \\
b & c & a
\end{bmatrix}$.

\begin{enumerate}
\item Show that $x^{3} - (a^{2} + b^{2} + c^{2})x - 2abc$ has real roots by considering $A$.

\item Show that $a^{2} + b^{2} + c^{2} \geq ab + ac + bc$ by considering $B$.
%\item  $c_{B}(x) = [x - (a + b + c)][x^{2} - k]$ where $k = a^{2} + b^{2} + c^{2} - [ab + ac + bc]$. Use Theorem~\ref{thm:016397}.
\end{enumerate}
\end{problem}

\begin{problem}\label{prob:2x2_special_nilpotent}
Assume the $2 \times 2$ matrix $A$ is similar to an upper triangular matrix. If $\mbox{tr} A = 0 = \mbox{tr} A^{2}$, show that $A^{2}$ is equal to the zero matrix.
\end{problem}

\begin{problem}\label{prob:2x2_similar_transpose}
Show that $A$ is similar to $A^{T}$ for all $2 \times 2$ matrices $A$. 
\begin{hint}
Let $A =\begin{bmatrix}
a & b \\
c & d
\end{bmatrix}$. If $c = 0$ treat the cases $b = 0$ and $b \neq 0$ separately. If $c \neq 0$, reduce to the case $c = 1$ using Exercise~\ref{prob:5_5_12}~\ref{prob:5_5_12d}.
\end{hint}
\end{problem}

\begin{problem}\label{prb:8.26} Suppose $A$ is an $n\times n$ matrix and let $\vec{v}$ be an
eigenvector such that $A\vec{v}=\lambda \vec{v}$. Also suppose the
characteristic polynomial of $A$ is
\begin{equation*}
\det \left( z I-A\right) =z ^{n}+a_{n-1} z ^{n-1}+\cdots
+a_{1}z +a_{0}
\end{equation*}
Explain why
\begin{equation*}
\left( A^{n}+a_{n-1}A^{n-1}+\cdots +a_{1}A+a_{0}I\right) \vec{v}=0
\end{equation*}
Use this to prove that the Cayley-Hamilton
theorem holds for any diagonalizable matrix $A$. (The Cayley-Hamilton theorem says that $A$ satisfies its
characteristic equation, i.e.,
\begin{equation*}
A^{n}+a_{n-1}A^{n-1}+\cdots +a_{1}A+a_{0}I=0
\end{equation*}.  (For a proof of the general case, see Theorem \ref{th:Cayley_Hamilton})
\end{problem}

\begin{problem}\label{prb:8.27} Suppose the characteristic polynomial of an $n\times n$ matrix $A$ is
$\lambda^{n}-1$. Find $A^{mn}$ where $m$ is an integer.

Click the arrow to see answer.
\begin{expandable}{}{}
The eigenvalues are distinct because
they are the $n^{th}$ roots of $1$. Hence if $\vec{x}$ is a given vector with
\[
\vec{x}=\sum_{j=1}^{n}a_{j}\vec{v}_{j}
\]
then
\[
A^{nm}\vec{x}=A^{nm}\sum_{j=1}^{n}a_{j}\vec{v}_{j}=
\sum_{j=1}^{n}a_{j}A^{nm}\vec{v}_{j}=\sum_{j=1}^{n}a_{j}\vec{v}_{j}=\vec{x}
\]
so $A^{nm}=I$.
\end{expandable}
\end{problem}



\section*{Bibliography}

These problems come from Chapter 7 of Ken Kuttler's \href{https://open.umn.edu/opentextbooks/textbooks/a-first-course-in-linear-algebra-2017}{\it A First Course in Linear Algebra}. (CC-BY)

Ken Kuttler, {\it  A First Course in Linear Algebra}, Lyryx 2017, Open Edition, pp. 359--401.

\end{document}