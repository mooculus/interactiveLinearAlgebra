\documentclass{ximera}
%% You can put user macros here
%% However, you cannot make new environments

\graphicspath{
    {./}
    {ABOUT/}
    {./LTR-0070/}
    {./VEC-0060/}
    {./APP-0020/}
    {./HowToUse/}
    {./APP-0070/}
    {./pictures/}
    {./PROB_linTrans}
    }

\usepackage{tikz}
\usepackage{tkz-euclide}
\usepackage{tikz-3dplot}
\usepackage{tikz-cd}
\usetikzlibrary{shapes.geometric}
\usetikzlibrary{arrows}
%\usetkzobj{all}
\pgfplotsset{compat=1.13} % prevents compile error.

%\renewcommand{\vec}[1]{\mathbf{#1}}
\renewcommand{\vec}{\mathbf}
\newcommand{\RR}{\mathbb{R}}
\newcommand{\dfn}{\textit}
\newcommand{\dotp}{\cdot}
\newcommand{\id}{\text{id}}
\newcommand\norm[1]{\left\lVert#1\right\rVert}
 
\newtheorem{general}{Generalization}
\newtheorem{initprob}{Exploration Problem}

\tikzstyle geometryDiagrams=[ultra thick,color=blue!50!black]

%\DefineVerbatimEnvironment{octave}{Verbatim}{numbers=left,frame=lines,label=Octave,labelposition=topline}



\usepackage{mathtools}


\title{Essential Vocabulary} \license{CC BY-NC-SA 4.0}



\begin{document}
\begin{abstract}
\end{abstract}
\maketitle


\begin{onlineOnly}
\section*{Essential Vocabulary}
Here is a  \href{https://quizlet.com/906039813/chapter-7-vocabulary-flash-cards/?i=y06sd&x=1jqt}{link to a list of these terms on Quizlet}
\end{onlineOnly}

\begin{tikzpicture}[scale=1]
   \filldraw[teal, opacity=0.3](0,0)--(20,0)--(20,0.1)--(0,0.1)--cycle;
 \end{tikzpicture}

Adjugate of a matrix (the term Adjoint is also sometimes used)
\begin{expandable}{}{}
    The transpose of the matrix of cofactors of a matrix - it is part of a formula for the inverse of a matrix.
\end{expandable}

\begin{tikzpicture}[scale=1]
   \filldraw[teal, opacity=0.3](0,0)--(20,0)--(20,0.1)--(0,0.1)--cycle;
 \end{tikzpicture}

Cofactor expansion
\begin{expandable}{}{}
    A method to compute $\det A$ using determinants of minor matrices associated with one row or one column.
\end{expandable}

\begin{tikzpicture}[scale=1]
   \filldraw[teal, opacity=0.3](0,0)--(20,0)--(20,0.1)--(0,0.1)--cycle;
 \end{tikzpicture}

Cramer's rule
\begin{expandable}{}{}
    A method of solving systems of equations that uses determinants.
\end{expandable}

\begin{tikzpicture}[scale=1]
   \filldraw[teal, opacity=0.3](0,0)--(20,0)--(20,0.1)--(0,0.1)--cycle;
 \end{tikzpicture}

Determinant
\begin{expandable}{}{}
    A function that assigns a scalar output to each square matrix $A$, denoted $\det A$ - it is nonzero if and only if $A$ is invertible.  Geometrically speaking, the determinant of a linear transformation of a square matrix is the factor by which area (or volume or hypervolume) is scaled by the transformation.
\end{expandable}

\begin{tikzpicture}[scale=1]
   \filldraw[teal, opacity=0.3](0,0)--(20,0)--(20,0.1)--(0,0.1)--cycle;
 \end{tikzpicture}

Laplace Expansion Theorem
\begin{expandable}{}{}
    The determinant of a matrix can be computed using cofactor expansion along ANY row or ANY column.
\end{expandable}

\begin{tikzpicture}[scale=1]
   \filldraw[teal, opacity=0.3](0,0)--(20,0)--(20,0.1)--(0,0.1)--cycle;
 \end{tikzpicture}

Properties of determinants
\begin{expandable}{}{}
\begin{enumerate}
    \item The determinant of a triangular matrix is the product of its diagonal entries.

    \item The determinant of a matrix is equal to the determinant of its transpose.

    \item The determinant of the inverse of a matrix is the reciprocal of the determinant of the matrix.

    \item A matrix with a zero row has determinant zero.

    \item Interchanging two rows of a matrix changes the sign of its determinant.

    \item A matrix with two identical rows has determinant zero.

    \item Multiplying a row of a matrix by $k$ multiplies the determinant by a factor of $k$.

    \item Multiplying a matrix by $k$ multiplies the determinant by a factor of $k^n$.

    \item Adding a multiple of one row of a matrix to another row does not change the determinant. 

    \item A matrix is singular if and only if its determinant is zero.

    \item The determinant of a product is equal to the product of the determinants.
\end{enumerate}
\end{expandable}

\begin{tikzpicture}[scale=1]
   \filldraw[teal, opacity=0.3](0,0)--(20,0)--(20,0.1)--(0,0.1)--cycle;
 \end{tikzpicture}

\end{document}


