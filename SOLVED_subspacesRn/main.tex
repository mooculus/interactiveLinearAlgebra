\documentclass{ximera}
%% You can put user macros here
%% However, you cannot make new environments

\graphicspath{
    {./}
    {ABOUT/}
    {./LTR-0070/}
    {./VEC-0060/}
    {./APP-0020/}
    {./HowToUse/}
    {./APP-0070/}
    {./pictures/}
    {./PROB_linTrans}
    }

\usepackage{tikz}
\usepackage{tkz-euclide}
\usepackage{tikz-3dplot}
\usepackage{tikz-cd}
\usetikzlibrary{shapes.geometric}
\usetikzlibrary{arrows}
%\usetkzobj{all}
\pgfplotsset{compat=1.13} % prevents compile error.

%\renewcommand{\vec}[1]{\mathbf{#1}}
\renewcommand{\vec}{\mathbf}
\newcommand{\RR}{\mathbb{R}}
\newcommand{\dfn}{\textit}
\newcommand{\dotp}{\cdot}
\newcommand{\id}{\text{id}}
\newcommand\norm[1]{\left\lVert#1\right\rVert}
 
\newtheorem{general}{Generalization}
\newtheorem{initprob}{Exploration Problem}

\tikzstyle geometryDiagrams=[ultra thick,color=blue!50!black]

%\DefineVerbatimEnvironment{octave}{Verbatim}{numbers=left,frame=lines,label=Octave,labelposition=topline}



\usepackage{mathtools}


\title{Solved Problems} \license{CC BY-NC-SA 4.0}

\begin{document}

\begin{abstract}
\end{abstract}
\maketitle

\section*{Solved Problems for Chapter 5}

\begin{problem}\label{prb:5.10} Let $M=\left\{ \vec{u}=\left[ \begin{array}{c}
u_{1} \\
u_{2} \\
u_{3} \\
u_{4}
\end{array}\right] \in
\mathbb{R}^{4}:|u_{1}| \leq 4\right\} .$ Is $M$ a
subspace? Explain.

Click the arrow to see answer.  
\begin{expandable}{}{}
No. $\left[
\begin{array}{r}
1 \\
0 \\
0 \\
0
\end{array}
\right]$ is in $M$ but $10\left[ \begin{array}{r}
1 \\
0 \\
0 \\
0
\end{array}
\right]$ is not in $M$.
\end{expandable}
\end{problem}

\begin{problem}\label{prb:5.11} Let $M=\left\{ \vec{u}=\left[ \begin{array}{c}
u_{1} \\
u_{2} \\
u_{3} \\
u_{4}
\end{array}\right] \in
\mathbb{R}^{4}:u_{i}\geq 0\text{ for each }i=1,2,3,4\right\} .$ Is $M$ a
subspace? Explain.

Click the arrow to see answer.  
\begin{expandable}{}{}
This is not a subspace. $\left[ \begin{array}{r}
1 \\
1 \\
1 \\
1
\end{array}
\right] $
is in $M$. However, $\left( -1\right) \left[
\begin{array}{r}
1 \\
1 \\
1 \\
1
\end{array}
\right] $ is not in $M$.
\end{expandable}
\end{problem}

\begin{problem}\label{prb:5.13} Let $\vec{w}\in \mathbb{R}^{4}$ and let $M=\left\{ \vec{u}
=\left[
\begin{array}{c}
u_{1} \\
u_{2} \\
u_{3} \\
u_{4}
\end{array}\right] \in \mathbb{R}^{4}:\vec{w}\dotp \vec{u}
=0\right\} .$ Is $M$ a subspace? Explain.

Click the arrow to see answer.  
\begin{expandable}{}{}
Yes, this is a subspace.  
First, $M$ is not empty because the zero vector is in $M$.  You can verify closure under vector addition and scalar multiplication using the properties of the dot product as follows.  If $\vec{u}$ is in $M$, then $\vec{w}\dotp\vec{u}=0$ and $\vec{w}\dotp k\vec{u}=0$ for all constants $k$.  Finally, if $\vec{u}_1$ and $\vec{u}_2$ are in $M$, then $\vec{w}\dotp(\vec{u}_1+\vec{u}_2)=\vec{w}\dotp\vec{u}_1+\vec{w}\dotp\vec{u}_2=0$.
\end{expandable}
\end{problem}

\begin{problem}\label{prb:5.15} Let $M=\left\{ \vec{u}=\left[
\begin{array}{c}
u_{1} \\
u_{2} \\
u_{3} \\
u_{4}
\end{array}\right] \in
\mathbb{R}^{4}:u_{3}=u_{1}=0\right\} .$ Is $M$ a subspace? Explain.

Click the arrow to see answer.  
\begin{expandable}{}{}
This is a subspace. $M$ is not empty because the zero vector is in it.  $M$ is closed  under vector addition and scalar
multiplication.
\end{expandable}
\end{problem}

\begin{problem}\label{prb:5.24} If you have $5$ vectors in $\mathbb{R}^{5}$ and the vectors are
linearly independent, can it always be concluded they span $\mathbb{R}^{5}?$
Explain.

Click the arrow to see answer.  
\begin{expandable}{}{}
 Yes. If not, there would exist a vector not in the span. But then
you could add in this vector and obtain a linearly independent set of
vectors with more vectors than a basis.
\end{expandable}
\end{problem}

\begin{problem}\label{prb:5.25} If you have $6$ vectors in $\mathbb{R}^{5},$ is it possible they are
linearly independent? Explain.

Click the arrow to see answer.  
\begin{expandable}{}{}
A matrix $A$ with five rows and six columns will contain a non-pivot column, giving rize to a non-trivial solution (infinitely many of them) to the equation $A\vec{x}=\vec{0}$.  This shows that the columns of $A$ are linearly dependent.
\end{expandable}
\end{problem}

\begin{problem}\label{prb:5.27} Suppose $V, W$ are subspaces of $\mathbb{R}^{n}.$ Let $V\cap W$
be all vectors which are in both $V$ and $W$. Show that $V \cap W$ is a subspace also.

Click the arrow to see answer.  
\begin{expandable}{}{}
If $\vec{x}, \vec{y}\in V\cap W,$ then for scalars $\alpha
,\beta ,$ the linear combination $\alpha \vec{x}+\beta \vec{y}$ must
be in both $V$ and $W$ since they are both subspaces.
\end{expandable}
\end{problem}

\begin{problem}\label{prb:5.37} Find the rank of the following matrix, and find a basis for the column spaces.
\begin{equation*}
\left[
\begin{array}{rrrr}
1 & 0 & 3 & 0 \\
3 & 1 & 10 & 0 \\
-1 & 1 & -2 & 1 \\
1 & -1 & 2 & -2
\end{array}
\right]
\end{equation*}

Click the arrow to see answer.  

\begin{expandable}{}{}
$$\text{rref}\left(\begin{bmatrix}
1 & 0 & 3 & 0 \\
3 & 1 & 10 & 0 \\
-1 & 1 & -2 & 1 \\
1 & -1 & 2 & -2
\end{bmatrix}\right)=\begin{bmatrix}1 &0 &3 &0\\
 0& 1& 1& 0\\
 0& 0& 0& 1\\
 0& 0& 0& 0\end{bmatrix}$$
 The rank of the matrix is 3.  We can use columns 1, 2, and 4 as a basis for the column space.
\end{expandable}
\end{problem}

\begin{problem}\label{prb:5.38a} Find a basis for $\mbox{null} \left(A \right)$.

$$A = \left[ \begin{array}{rrr}
1 & 0 & -1 \\
-1 & 1 & 3 \\
3 & 2 & 1
\end{array} \right]$$

Click the arrow to see answer.

\begin{expandable}{}{}
    $$\text{rref}\left(\left[ \begin{array}{rrr}
1 & 0 & -1 \\
-1 & 1 & 3 \\
3 & 2 & 1
\end{array} \right]\right)=\begin{bmatrix}
    1& 0& -1\\
 0 &1& 2\\
 0& 0& 0
\end{bmatrix}$$

A basis for the null space is $\left\{\begin{bmatrix}
     1\\-2\\1
 \end{bmatrix}\right\}$.
\end{expandable}
\end{problem}

\begin{problem}\label{prb:5.38} Find a basis for $\mbox{null} \left(A \right)$
$$ A = \left[ \begin{array}{rrrr}
2 & -1 & 3 & 5 \\
2 & 0 & 1 & 2 \\
6 & 4 & -5 & -6 \\
0 & 2 & -4 & -6
\end{array} \right]$$

Click the arrow to see answer.

\begin{expandable}{}{}
 $$\text{rref}\left(\left[ \begin{array}{rrrr}
2 & -1 & 3 & 5 \\
2 & 0 & 1 & 2 \\
6 & 4 & -5 & -6 \\
0 & 2 & -4 & -6
\end{array} \right]\right)=
 \begin{bmatrix} 1 &0 &1/2 &1\\
 0 &1 &-2 &-3\\
 0& 0& 0& 0\\
 0& 0& 0& 0\end{bmatrix}
 $$  A basis for the null space is $\left\{\begin{bmatrix}
     -1\\3\\0\\1
 \end{bmatrix}, \begin{bmatrix}
     -1/2\\2\\1\\0
 \end{bmatrix}\right\}$.
\end{expandable}
\end{problem}

\begin{problem}\label{prob:nullZero}
Suppose matrix $A$ has linearly independent columns.  What is the null space of $A$?

Click the arrow to see answer.

\begin{expandable}{}{}
    The only solution to $A\vec{x}=\vec{0}$ is the trivial one.  Therefore, the only vector in $\text{null}(A)$ is the zero vector.
\end{expandable}
\end{problem}

\begin{problem}\label{prob:dimCollectionVectors}
    Let $V=\text{span}\left(\begin{bmatrix}2\\-1\\4\end{bmatrix}, \begin{bmatrix}-1\\1\\3\end{bmatrix}, \begin{bmatrix}4\\-3\\-2\end{bmatrix}\right) $.  Find a basis for $V$.  What is the dimension of $V$?

Click the arrow to see answer.

\begin{expandable}{}{}
    Form a matrix $A$ using the given vectors as columns.  $\text{rref}(A)=\begin{bmatrix}1& 0 &1\\
 0& 1& -2\\
 0& 0& 0\end{bmatrix}$.  The leading $1$'s are in the first two columns.  We can use the first two of the given vectors to form a basis of $V$.  Therefore $\text{dim}(V)=2$.
\end{expandable}
\end{problem}


\section*{Bibliography}
Some of the problems come from the end of Chapter 4 of Ken Kuttler's \href{https://open.umn.edu/opentextbooks/textbooks/a-first-course-in-linear-algebra-2017}{\it A First Course in Linear Algebra}. (CC-BY)

Ken Kuttler, {\it  A First Course in Linear Algebra}, Lyryx 2017, Open Edition, pp. 227--232.

\end{document}