\documentclass{ximera}
%% You can put user macros here
%% However, you cannot make new environments

\graphicspath{
    {./}
    {ABOUT/}
    {./LTR-0070/}
    {./VEC-0060/}
    {./APP-0020/}
    {./HowToUse/}
    {./APP-0070/}
    {./pictures/}
    {./PROB_linTrans}
    }

\usepackage{tikz}
\usepackage{tkz-euclide}
\usepackage{tikz-3dplot}
\usepackage{tikz-cd}
\usetikzlibrary{shapes.geometric}
\usetikzlibrary{arrows}
%\usetkzobj{all}
\pgfplotsset{compat=1.13} % prevents compile error.

%\renewcommand{\vec}[1]{\mathbf{#1}}
\renewcommand{\vec}{\mathbf}
\newcommand{\RR}{\mathbb{R}}
\newcommand{\dfn}{\textit}
\newcommand{\dotp}{\cdot}
\newcommand{\id}{\text{id}}
\newcommand\norm[1]{\left\lVert#1\right\rVert}
 
\newtheorem{general}{Generalization}
\newtheorem{initprob}{Exploration Problem}

\tikzstyle geometryDiagrams=[ultra thick,color=blue!50!black]

%\DefineVerbatimEnvironment{octave}{Verbatim}{numbers=left,frame=lines,label=Octave,labelposition=topline}



\usepackage{mathtools}


\title{Essential Vocabulary} \license{CC BY-NC-SA 4.0}



\begin{document}
\begin{abstract}
\end{abstract}
\maketitle


\begin{onlineOnly}
\section*{Essential Vocabulary}
\end{onlineOnly}

Coordinate vector

\begin{expandable}{}{}
    Let $V$ be a vector space, and let $\mathcal{B}=\{\vec{v}_1, \ldots ,\vec{v}_n\}$ be a basis for $V$.  If $\vec{v}=a_1\vec{v}_1+\ldots +a_n\vec{v}_n$, then the vector in $\RR^n$ whose components are the coefficients $a_1, \ldots ,a_n$  is said to be the \dfn{coordinate vector for $\vec{v}$ with respect to $\mathcal{B}$}.  We denote the coordinate vector by $[\vec{v}]_{\mathcal{B}}$ and write:
$$[\vec{v}]_{\mathcal{B}}=\begin{bmatrix}a_1\\\vdots \\a_n\end{bmatrix}$$
\end{expandable}

\begin{tikzpicture}[scale=1]
   \filldraw[teal, opacity=0.3](0,0)--(20,0)--(20,0.1)--(0,0.1)--cycle;
 \end{tikzpicture}

Finite-dimensional vector space

\begin{expandable}{}{}
    A vector space is said to be \dfn{finite-dimensional} if it is spanned by finitely many vectors.
\end{expandable}

\begin{tikzpicture}[scale=1]
   \filldraw[teal, opacity=0.3](0,0)--(20,0)--(20,0.1)--(0,0.1)--cycle;
 \end{tikzpicture}

 Isomorphism

 \begin{expandable}{}{}
     Let $V$ and $W$ be vector spaces.  If there exists an invertible linear transformation $T:V\rightarrow W$ we say that $V$ and $W$ are \dfn{isomorphic} and write $V\cong W$.  The invertible linear transformation $T$ is called an \dfn{isomorphism}.
 \end{expandable}

 \begin{tikzpicture}[scale=1]
   \filldraw[teal, opacity=0.3](0,0)--(20,0)--(20,0.1)--(0,0.1)--cycle;
 \end{tikzpicture}

 Matrix of a linear transformation

 \begin{expandable}{}{}
     Let $V$ and $W$ be finite-dimensional vector spaces with ordered bases $\mathcal{B}=\{\vec{v}_1,\vec{v}_2,\ldots,\vec{v}_n\}$ and $\mathcal{C}$, respectively.  Suppose $T:V\rightarrow W$ is a linear transformation.  
$$\text{Let}\quad A=\begin{bmatrix}
           | & |& &|\\
		[T(\vec{v}_1)]_{\mathcal{C}} & [T(\vec{v}_2)]_{\mathcal{C}}&\dots &[T(\vec{v}_n)]_{\mathcal{C}}\\
		|&| & &|
         \end{bmatrix}$$
Then $A[\vec{v}]_{\mathcal{B}}=[T(\vec{v})]_{\mathcal{C}}$ for all vectors $\vec{v}$ in $V$.  

Matrix $A$ is called the matrix of $T$ with respect to ordered bases $\mathcal{B}$ and $\mathcal{C}$.
 \end{expandable}

 \begin{tikzpicture}[scale=1]
   \filldraw[teal, opacity=0.3](0,0)--(20,0)--(20,0.1)--(0,0.1)--cycle;
 \end{tikzpicture}

 One-to-one linear transformation

 \begin{expandable}{}{}
     A linear transformation $T:V\rightarrow W$ is \dfn{one-to-one} if 
$$T(\vec{v}_1)=T(\vec{v}_2)\quad \text{implies that}\quad \vec{v}_1=\vec{v}_2$$
 \end{expandable}

\begin{tikzpicture}[scale=1]
   \filldraw[teal, opacity=0.3](0,0)--(20,0)--(20,0.1)--(0,0.1)--cycle;
 \end{tikzpicture}

 Onto linear transformation

 \begin{expandable}{}{}
     A linear transformation $T:V\rightarrow W$ is \dfn{onto} if for every element $\vec{w}$ of $W$, there exists an element $\vec{v}$ of $V$ such that $T(\vec{v})=\vec{w}$.
 \end{expandable}

 \begin{tikzpicture}[scale=1]
   \filldraw[teal, opacity=0.3](0,0)--(20,0)--(20,0.1)--(0,0.1)--cycle;
 \end{tikzpicture}

Subspace

\begin{expandable}{}{}
    A nonempty subset $U$ of a vector space $V$ is called a \dfn{subspace} of $V$, provided that $U$ is itself a vector space when given the same addition and scalar multiplication as $V$.
\end{expandable}

\begin{tikzpicture}[scale=1]
   \filldraw[teal, opacity=0.3](0,0)--(20,0)--(20,0.1)--(0,0.1)--cycle;
 \end{tikzpicture}

Vector Space

\begin{expandable}{}{}
     Let $V$ be a nonempty set.  Suppose that elements of $V$ can be added together and multiplied by scalars.  The set $V$, together with operations of addition and scalar multiplication, is called a \dfn{vector space} provided that 
  \begin{itemize}
  \item[] $V$ is closed under addition
  \item[] $V$ is closed under scalar multiplication
  \end{itemize}
  and the following properties hold for $\vec{u}$, $\vec{v}$ and $\vec{w}$ in $V$ and scalars $k$ and $p$:
  \begin{enumerate}
   \item 
  Commutative Property of Addition:\quad
  $\vec{u}+\vec{v}=\vec{v}+\vec{u}$
  \item 
  Associative Property of Addition:\quad
  $(\vec{u}+\vec{v})+\vec{w}=\vec{u}+(\vec{v}+\vec{w})$
  \item 
  Existence of Additive Identity:\quad
  $\vec{u}+\vec{0}=\vec{u}$
  \item 
  Existence of Additive Inverse:\quad
  $\vec{u}+(-\vec{u})=\vec{0}$
  \item 
  Distributive Property over Vector Addition:\quad
  $k(\vec{u}+\vec{v})=k\vec{u}+k\vec{v}$
  \item 
  Distributive Property over Scalar Addition:\quad
  $(k+p)\vec{u}=k\vec{u}+p\vec{u}$
  \item 
  Associative Property for Scalar Multiplication:\quad
  $k(p\vec{u})=(kp)\vec{u}$
  \item 
  Multiplication by $1$:\quad
  $1\vec{u}=\vec{u}$
  \end{enumerate}
We will refer to elements of $V$ as \dfn{vectors}.
\end{expandable}

\begin{tikzpicture}[scale=1]
   \filldraw[teal, opacity=0.3](0,0)--(20,0)--(20,0.1)--(0,0.1)--cycle;
 \end{tikzpicture}

 Note that definitions of span, linear independence, basis, and dimension are analogous to those for subspaces of $\RR^n$.

 

\end{document}
