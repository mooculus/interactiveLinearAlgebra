\documentclass{ximera}
\input{../preamble.tex}

\title{Octave for Chapter 7} \license{CC BY-NC-SA 4.0}

\begin{document}

\begin{abstract}
\end{abstract}
\maketitle

\section*{Octave for Chapter 7}

The templates in this section provide sample Octave code for finding subspaces associated with matrices as well as the kernel and image of a linear transformation. You can access our code through the link at the bottom of each template.  Feel free to modify the code and experiment to learn more!  

You can write your own code using Octave software or online Octave cells.  To access Octave cells online, go to the \href{https://sagecell.sagemath.org/}{Sage Math Cell Webpage}, select OCTAVE as the language, enter your code, and press EVALUATE.  

To ''save" or share your online code, click on the \emph{Share} button, select \emph{Permalink}, then copy the address directly from the browser window.  You can store this link to access your work later or share this link with others.  You will need to get a new Permalink every time you modify the code.

\subsection*{Octave Tutorial}
\begin{warning}
Computing the determinant is extremely inefficient.  As a result, determinants are rarely used in computational practice.  Examples and exercises in this section are designed to illustrate its theoretical properties and should not be taken as a templates for large-scale computational problems.
\end{warning}

We will use the \emph{det} function to find determinants in Octave.

\begin{example}
  Deterimine wheather each of the following matrices is singular.
  $$A=\begin{bmatrix}-1 & 5 & -1\\7 & -4 & -1\\5 & 6 & -3\end{bmatrix},\quad B=\begin{bmatrix}6 & -1 & 2\\12 & -3 & 1\\-1 & 4 & 11\end{bmatrix}$$
  \begin{explanation}
  We find the determinant of each matrix.
  \begin{verbatim}
    % Define matrices A and B
    A=[-1  5  -1;
    7  -4  -1;
    5  6 -3];

    B=[6  -1  2;
    12  -3  1;
    -1  4  11];

    % Find det(A) and det(B)
    det(A)
    det(B)
  \end{verbatim}

\href{https://sagecell.sagemath.org/?z=eJwtjTsKgDAQRPtA7jCNYIoU67cQiwTxEsEiaIQUWqj3x42xmreP2d0CU9jjGXD454pruGHgzw1WCjM6TUALaBqk6Dmbn9l10PXCKIUdXZc8UPFMFXMNpFpyvEKUiwXmyKe38JRGfV8SWiVFVjmtegFjQB_C&lang=octave&interacts=eJyLjgUAARUAuQ==}{Link to code}.

We see that $\text{det}(A)=0$ and $\text{det}(B)\neq 0$.  We conclude that $A$ is singular and $B$ is not.

  \end{explanation}
\end{example}

\begin{problem}\label{prob_oct_det1}
Determine whether each of the following matrices is singular. (Try at least two different approaches.)  If $A$ is non-singular, find $A^{-1}$, $\det(A)$, and $\det(A^{-1})$.  Illustrate the relationship between $\det(A)$ and $\det(A^{-1})$ computationally.
\begin{enumerate}
    \item $$A=\begin{bmatrix}1 &-2 & 1 & 6 & 1\\0 & -1 & 3 & 5 & -1\\2 & -1 & 8 & 4 & 2\\0 & 14 & -11 & -7 & 2\\3 & -3 & 9 & 10 & 3\end{bmatrix}$$
    \item $$A=\begin{bmatrix}1 &-2 & 1 & 6 &1\\0 & -1 & 3 & 5 & 0\\2 & -1 & 8 & 4 & 1\\3 & 12 & -2 & 3 & 0\\2 & 0 & -11 & -9 & 5\end{bmatrix}$$
\end{enumerate}



  \end{problem}
\end{document}