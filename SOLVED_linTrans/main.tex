\documentclass{ximera}
%% You can put user macros here
%% However, you cannot make new environments

\graphicspath{
    {./}
    {ABOUT/}
    {./LTR-0070/}
    {./VEC-0060/}
    {./APP-0020/}
    {./HowToUse/}
    {./APP-0070/}
    {./pictures/}
    {./PROB_linTrans}
    }

\usepackage{tikz}
\usepackage{tkz-euclide}
\usepackage{tikz-3dplot}
\usepackage{tikz-cd}
\usetikzlibrary{shapes.geometric}
\usetikzlibrary{arrows}
%\usetkzobj{all}
\pgfplotsset{compat=1.13} % prevents compile error.

%\renewcommand{\vec}[1]{\mathbf{#1}}
\renewcommand{\vec}{\mathbf}
\newcommand{\RR}{\mathbb{R}}
\newcommand{\dfn}{\textit}
\newcommand{\dotp}{\cdot}
\newcommand{\id}{\text{id}}
\newcommand\norm[1]{\left\lVert#1\right\rVert}
 
\newtheorem{general}{Generalization}
\newtheorem{initprob}{Exploration Problem}

\tikzstyle geometryDiagrams=[ultra thick,color=blue!50!black]

%\DefineVerbatimEnvironment{octave}{Verbatim}{numbers=left,frame=lines,label=Octave,labelposition=topline}



\usepackage{mathtools}


\title{Solved Problems} \license{CC BY-NC-SA 4.0}

\begin{document}

\begin{abstract}
\end{abstract}
\maketitle

\section*{Solved Problems for Chapter 6}

\begin{problem}\label{prb:6.2} Show that the function $T_{\vec{u}}$ defined by $T_{\vec{u}}
\left( \vec{v}\right) = \vec{v}-\mbox{proj}_{\vec{u}}\left(
\vec{v}\right) $ is also a linear transformation.

Click on the arrow to see answer.

\begin{expandable}{}{}
$$
T_{\vec{u}}\left( a\vec{v}+b\vec{w}\right) =a\vec{v}+b\vec{w}-\frac{\left( a\vec{v}+b\vec{w}\right)\dotp \vec{u} }{\norm{ \vec{u}
} ^{2}}\vec{u} =$$
$$
=a\vec{v}-a\frac{\left( \vec{v}\dotp \vec{u}\right) }{\norm{ \vec{u} } ^{2}}\vec{u}+b\vec{w}-b\frac{\left( \vec{w}\dotp \vec{u}
\right) }{\norm{ \vec{u}} ^{2}}\vec{u}
=aT_{\vec{u}}\left( \vec{v}\right) +bT_{\vec{u}}\left( \vec{w}
\right)
$$
\end{expandable}
\end{problem}

\begin{problem}\label{prb:6.3} Let $\vec{u}$ be a fixed vector. The function
$T_{\vec{u}}$ defined by $T_{\vec{u}}\vec{v}=\vec{u}+\vec{v}$ has the effect of
translating all vectors by adding $\vec{u}\neq \vec{0}$. Show that $T_{\vec{u}}$ is not a
linear transformation. 

Click the arrow to see answer.

\begin{expandable}{}{}
Linear
transformations take $\vec{0}$ to $\vec{0}$ which $T$ does not. Also $T_{\vec{a}}\left( \vec{u}+\vec{v}\right) \neq T_{\vec{a}}\vec{u}+T_{\vec{a}}
\vec{v}$.
\end{expandable}
\end{problem}

\begin{problem}\label{prb:6.11}  Find the matrix for the linear transformation which
rotates every vector in $\mathbb{R}^{2}$ through an angle of $\pi /3.$

Click on the arrow to see answer.
\begin{expandable}{}{}
$$\left[
\begin{array}{cc}
\cos \left(
\frac{\pi }{3}\right) & -\sin \left( \frac{\pi }{3}\right) \\
\sin \left( \frac{\pi }{3}\right) & \cos \left( \frac{\pi }{3}\right)%
\end{array}
\right] = \left[
\begin{array}{cc}
\frac{1}{2} & -\frac{1}{2}\sqrt{3} \\
\frac{1}{2}\sqrt{3} & \frac{1}{2}
\end{array}
\right] $$
\end{expandable}
\end{problem}

\begin{problem}\label{prb:6.16} Find the matrix for the linear transformation which rotates every
vector in $\mathbb{R}^{2}$ through an angle of $2\pi /3$ and then reflects
across the $x$ axis.

Click on the arrow to see answer.
\begin{expandable}{}{}
\[
\left[
\begin{array}{rr}
1 & 0 \\
0 & -1
\end{array}
\right] \left[
\begin{array}{cc}
\cos \left( \frac{2\pi }{3}\right)  & -\sin \left( \frac{2\pi }{3}\right)
\\
\sin \left( \frac{2\pi }{3}\right)  & \cos \left( \frac{2\pi }{3}\right)
\end{array}
\right] = \left[
\begin{array}{cc}
-\frac{1}{2} & -\frac{1}{2}\sqrt{3} \\
-\frac{1}{2}\sqrt{3} & \frac{1}{2}
\end{array}
\right]
\]
\end{expandable}
\end{problem}

\begin{problem}\label{prb:6.19} Find the matrix for the linear transformation which rotates every
vector in $\mathbb{R}^{2}$ through an angle of $\pi /6$ and then reflects
across the $x$ axis followed by a reflection across the $y$ axis.

Click on the arrow to see answer.
\begin{expandable}{}{}
\[
\left[
\begin{array}{rr}
-1 & 0 \\
0 & 1
\end{array}
\right] \left[
\begin{array}{cc}
\cos \left( \frac{\pi }{6}\right)  & -\sin \left( \frac{\pi }{6}\right)  \\
\sin \left( \frac{\pi }{6}\right)  & \cos \left( \frac{\pi }{6}\right)
\end{array}
\right] = \left[
\begin{array}{cc}
-\frac{1}{2}\sqrt{3} & \frac{1}{2} \\
\frac{1}{2} & \frac{1}{2}\sqrt{3}
\end{array}
\right]
\]
\end{expandable}
\end{problem}

\begin{problem}\label{prb:6.5} Let $T$ be a linear transformation induced by the matrix $A = \left[ \begin{array}{rr}
3 & 1 \\
-1 & 2
\end{array}\right]$ and let $S$ be a linear transformation induced by $B = \left[ \begin{array}{rr}
0 & -2 \\
4 & 2
\end{array}\right]$. Find matrix of $S \circ T$ and find $\left( S \circ T \right) \left( \vec{x} \right)$ for $\vec{x} = \left[ \begin{array}{r}
2 \\
-1
\end{array} \right]$.

Click on the arrow to see answer.
\begin{expandable}{}{}
The matrix of $S \circ T$ is given by $BA$.
\[
\left[ \begin{array}{rr}
0 & -2 \\
4 & 2
\end{array}\right] \left[ \begin{array}{rr}
3 & 1 \\
-1 & 2
\end{array}\right] = \left[
\begin{array}{rr}
2 & -4 \\
10 & 8
\end{array}
\right]
\]
Now, $\left( S \circ T \right) \left( \vec{x} \right) = (BA) \vec{x}$.
\[
 \left[
\begin{array}{rr}
2 & -4 \\
10 & 8
\end{array}
\right]
\left[ \begin{array}{r}
2 \\
-1
\end{array} \right]
=
\left[
\begin{array}{r}
8 \\
12
\end{array}
\right]
\]

\end{expandable}
\end{problem}

\begin{problem}\label{prb:6.6} Let $T$ be a linear transformation and suppose $T \left( \left[ \begin{array}{r}
1 \\
-4
\end{array} \right] \right) = \left[ \begin{array}{r}
2 \\
-3
\end{array} \right]$. Suppose $S$ is a linear transformation induced by the matrix $B = \left[ \begin{array}{rr}
1 & 2 \\
-1 & 3
\end{array} \right]$. Find $\left( S \circ T \right) \left( \vec{x} \right)$ for $\vec{x} = \left[ \begin{array}{r}
1 \\
-4
\end{array} \right]$.

Click on the arrow to see answer.
\begin{expandable}{}{}
To find $\left( S \circ T \right) \left( \vec{x} \right)$ we compute $S(T(\vec{x}))$.
\[
\left[ \begin{array}{rr}
1 & 2 \\
-1 & 3
\end{array} \right]
\left[ \begin{array}{r}
2 \\
-3
\end{array} \right]
 = \left[
\begin{array}{r}
-4 \\
-11
\end{array}
\right]
\]
\end{expandable}
\end{problem}

\begin{problem}\label{prb:6.28}
 Let $T$ be a linear transformation given by
\[
T \left[ \begin{array}{r}
x\\
y
\end{array}\right] = \left[ \begin{array}{rrr}
1 &1  \\
1 & 1
\end{array}\right]
\left[ \begin{array}{r}
x\\
y
\end{array}\right]
\]
Find a basis for $\ker \left( T\right)$ and a basis for $\mbox{im} \left( T\right) $. Find the dimension of the kernel and the image  of $T$.

Click on the arrow to see answer.
\begin{expandable}{}{}
A basis for $\ker \left( T\right)$ is
$\left\{ \left[
\begin{array}{r}
1 \\
-1
\end{array}
\right] \right\}$
and a basis for $\mbox{im} \left( T\right)$ is
$\left\{ \left[
\begin{array}{r}
1 \\
1
\end{array}
\right] \right\}$. \\
There are many other possibilities for the specific bases. $\dim \left( \ker \left( T\right) \right)=1 $ and $\dim \left( \mbox{im} \left( T\right) \right)=1$.
\end{expandable}

\end{problem}


\begin{problem}\label{prb:6.29}
 Let $T$ be a linear transformation given by
\[
T \left[ \begin{array}{r}
x\\
y
\end{array}\right] = \left[ \begin{array}{rrr}
1 & 0  \\
1 & 1
\end{array}\right]
\left[ \begin{array}{r}
x\\
y
\end{array}\right]
\]
Find a basis for $\ker \left( T\right)$ and a basis for $\mbox{im}
\left( T\right) $.

Click on the arrow to see answer.

\begin{expandable}{}{}
In this case $\ker \left( T\right) =\{0\}$
and $\mbox{im} \left( T\right) = \mathbb{R}^2$ (pick any basis of $\RR^2$).
\end{expandable}

\end{problem}

\begin{problem}\label{prb:6.31}
 Let $T$ be a linear transformation given by
\[
T \left[ \begin{array}{r}
x\\
y \\
z
\end{array}\right] = \left[ \begin{array}{rrr}
1 & 1 & 1 \\
1 & 1 & 1
\end{array}\right]
\left[ \begin{array}{r}
x\\
y \\
z
\end{array}\right]
\]
What is $\dim  ( \ker \left( T \right) )$?

Click on the arrow to see answer.
\begin{expandable}{}{}
We can easily see that $\dim  ( \mbox{im} \left( T \right) ) =1$, and thus
$\dim  ( \ker \left( T \right) ) = 3 - \dim  ( \mbox{im} \left( T \right) ) = 3- 1 = 2$.
\end{expandable}
\end{problem}

\begin{problem}\label{prob:imOfVector1}
    Suppose $T:\RR^2\rightarrow\RR^3$ is a linear transformation such that $T\left(\begin{bmatrix}2\\-1\end{bmatrix}\right)=\begin{bmatrix}7\\-6\\1\end{bmatrix}$, and $T\left(\begin{bmatrix}4\\1\end{bmatrix}\right)=\begin{bmatrix}-2\\5\\0\end{bmatrix}$.  Find the image of $\begin{bmatrix}-2\\-2\end{bmatrix}$ under $T$.

    Click the arrow to see answer.

    \begin{expandable}{}{}
        First, observe that $\begin{bmatrix}-2\\-2\end{bmatrix}=\begin{bmatrix}2\\-1\end{bmatrix}-\begin{bmatrix}4\\1\end{bmatrix}$.  Therefore,
        $$T\left(\begin{bmatrix}-2\\-2\end{bmatrix}\right)=T\left(\begin{bmatrix}2\\-1\end{bmatrix}\right)-T\left(\begin{bmatrix}4\\1\end{bmatrix}\right)=\begin{bmatrix}7\\-6\\1\end{bmatrix}-\begin{bmatrix}-2\\5\\0\end{bmatrix}=\begin{bmatrix}9\\-11\\1\end{bmatrix}$$
    \end{expandable}
\end{problem}

\begin{problem}\label{prob:invImageOfVect}
    Suppose $T:\RR^2\rightarrow \RR^2$ is a linear transformation that maps $\vec{i}$ to $\begin{bmatrix}1\\1\end{bmatrix}$, and $\vec{j}$ to $\begin{bmatrix}-2\\3\end{bmatrix}$.  Find vector $\vec{v}$ such that $T(\vec{v})=\begin{bmatrix}20\\-30\end{bmatrix}$.

    Click the arrow to see answer.

    \begin{expandable}{}{}
        Observe that $T$ is induced by the matrix $A=\begin{bmatrix}1 & -2\\1 & 3\end{bmatrix}$.  We are looking for $\vec{v}$ such that $\begin{bmatrix}1 & -2\\1 & 3\end{bmatrix}\vec{v}=\begin{bmatrix}20\\-30\end{bmatrix}$.  Multiplying both sides by $A^{-1}$ gives us
        $$\vec{v}=\begin{bmatrix}
            0.6 & 0.4\\
 -0.2 & 0.2
        \end{bmatrix}\begin{bmatrix}20\\-30\end{bmatrix}=\begin{bmatrix}0\\-10\end{bmatrix}$$
    \end{expandable}
\end{problem}




\section*{Bibliography}
Some of the problems come from Chapter 5 of Ken Kuttler's \href{https://open.umn.edu/opentextbooks/textbooks/a-first-course-in-linear-algebra-2017}{\it A First Course in Linear Algebra}. (CC-BY)

Ken Kuttler, {\it  A First Course in Linear Algebra}, Lyryx 2017, Open Edition, pp. 272--315. 

\end{document}